\documentclass[UTF8]{ctexart}
%为中英混排,自动检测系统,自动适应合适字体
%确定编写文档类型为ctexart(ctex article),类似的还有ctexrep(ctex report)支持中文的report类
%
%ctexart类 包含如下五个控制序列定义行文组织结构
%\section{·}
%\subsection{·}
%\subsubsection{·}
%\paragraph{·}
%\subparagraph{·}


%旧兼容中英混排方法
%\documentclass{article}
%\usepackage{xeCJK}  %调用xeCJK宏包
%\setCJKmainfont{STFangSong}  %设置CJK字体为仿宋 搜索字体册


%=================================================================
						%导言区
%=================================================================

%添加amsmath宏包,调用数学公式
\usepackage{amsmath}
%添加graphicx宏包,调用图片
\usepackage{graphicx}
%控制页边距
\usepackage{geometry}
%\geometry{papersize={20cm,60cm}}
\geometry{left=4cm,right=4cm,top=2cm,bottom=4cm}
%页眉页脚
\usepackage{fancyhdr}
\pagestyle{fancy}
\lhead{\author}
\chead{\date}
\rhead{152xxxxxxxx}
\lfoot{}
\cfoot{\thepage}
\rfoot{}
\renewcommand{\headrulewidth}{0.4pt}
\renewcommand{\headwidth}{\textwidth}
\renewcommand{\footrulewidth}{0pt}


%全局定义
\title{中山大学数据科学与计算机学院本科生实验报告}
\author{詹宗沅}
\date{\today}



%=================================================================
						%正文显示区
%=================================================================
\begin{document}

%控制序列,可以将定义在导言区的信息按照一定格式展示出来
\maketitle
%控制序列,显示文章目录
\tableofcontents


\section{section}
this is content under section
  \subsection{subsection}
  this is content under subsection
    \subsubsection{susubsection}
    hei, i am subsubsection
      \paragraph{paragraph}
      paragraph is here, too
        \subparagraph{subparagraph}
        call me subparagrah


\section{math}
%数学公式  \usepackage{amsmath}
%行内:$...$    \[...\]
%行间:\begin{equation}  \end{equation}, equation在这里是个环境
假设一个,$Y_{eat_{who}}$所以有
\begin{equation}
f(x) \Rightarrow g(x)
\end{equation}
%行内公式也可以使用 \(...\) 或者 \begin{math} ... \end{math} 来插入,但略显麻烦。
%无编号的行间公式也可以使用 \begin{displaymath} ... \end{displaymath} 或者 \begin{equation*} ... \end{equation*} 来插入,但略显麻烦。(equation* 中的 * 表示环境不编号)
%也有 plainTeX 风格的 $$ ... $$ 来插入不编号的行间公式。但是在 LaTeX 中这样做会改变行文的默认行间距,不推荐。请参考我的回答。
  \subsection{界定符号}
  \[ \Biggl\langle\biggl\langle\Bigl\langle\bigl\langle\langle x
  \rangle\bigr\rangle\Bigr\rangle\biggr\rangle\Biggr\rangle \]

  \[ \Biggl\lvert\biggl\lvert\Bigl\lvert\bigl\lvert\lvert x
  \rvert\bigr\rvert\Bigr\rvert\biggr\rvert\Biggr\rvert \]

  \[ \Biggl\lVert\biggl\lVert\Bigl\lVert\bigl\lVert\lVert x
  \rVert\bigr\rVert\Bigr\rVert\biggr\rVert\Biggr\rVert \]
  %各种括号用 (), [], \{\}, \langle\rangle 等命令表示;注意花括号通常用来输入命令和环境的参数,所以在数学公式中它们前面要加 \。因为 LaTeX 中 | 和 \| 的应用过于随意,amsmath 宏包推荐用 \lvert\rvert 和 \lVert\rVert 取而代之。
  %为了调整这些分隔符的大小,amsmath宏包推荐使用 \big, \Big, \bigg, \Bigg 等一系列命令放在上述括号前面调整大小。


\section{图片和表格}
  \subsection{图片}
  %需要引入宏包grapicx,加入可选参数控制使得占页面大小的20%,包括其他参数
  \includegraphics[width = .2\textwidth]{Screenshot_1512636581.png}
  \subsection{表格}
  %表格,使用控制序列\begin{tabular} 环境,tabular这里是个环境,这里| l | c | r |控制每列的对齐方向
  %\hline 表示横线,& 分隔列,| 表示竖线,\\ 表示一行结束
  \begin{tabular}{| l | c | r |}   % 这里| l | c | r | 表示每行的样式, 包括||
    \hline
    dog & cat & fish \\
    \hline
    eat & die & swim \\
    \hline
    eat & die & swim \\
    \hline
  \end{tabular}


\section{浮动体}
%插图和表格通常需要占据大块空间,所以在文字处理软件中我们经常需要调整他们的位置。figure 和 table 环境可以自动完成这样的任务;这种自动调整位置的环境称作浮动体(float)。我们以 figure 为例。
%htbp 控制插图的位置, here   top   bottom   float page(Screenshot_1512636581.png)
\begin{figure}[htbp]
\centering   %使得插图居中
\includegraphics[width = .4\textwidth]{Screenshot_1512636581.png}   
\caption{实验截图}  %插图标题(标注),latex会自动加上编号
\label{fig:myphoto}  %label应该放在图片之后
\end{figure}

\end{document}









